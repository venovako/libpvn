A similar recursive approach would work for the $\ell^p$ norm of
${\bf x}$, where $p\ge 1$ and
$$\|{\bf x}\|_p^{}=\left(\sum_{i=1}^n|x_i^{}|^p\right)^{1/p},$$
if a function ${\rm norm}p(x,y)$ were available, where
$${\rm norm}p(x,y)=\left(|x|^p+|y|^p\right)^{1/p},$$
such that unwarranted under/over-flows are avoided, and relative
accuracy and monotonicity with respect to $|x|$ and $|y|$ are
guaranteed.

For $p=1$, ${\rm norm}1(x,y)=|x|+|y|$, for $p=2$,
${\rm norm}2(x,y)={\rm hypot}(x,y)$.  Else, as the first attempt,
$${\rm norm}p(x,y)=M\left(1+q^p\right)^{1/p},\quad
M=\max\{|x|,|y|\},\quad
m=\min\{|x|,|y|\},\quad
q=m/M \hbox{ or 0 if } m=0.$$
Let $z=q^{p/2}$.  Then $1+q^p=1+z^2\approx{\rm fma}(z,z,1)$.
With ${\rm cr\_pow}$, the algorithm might be as follows:
$$X={\rm fabs}(x),\quad Y={\rm fabs}(y);$$
$$M={\rm fmax}(X,Y),\quad m={\rm fmin}(X,Y);$$
$$q=m/M,\quad Q={\rm fmax}(q,0);$$
$$z={\rm cr\_pow}(Q,p/2),\quad Z={\rm fma}(z,z,1);$$
$$C={\rm cr\_pow}(Z,1/p);$$
$${\rm norm}p(x,y)\approx M\cdot C.$$

\hrule\vskip+1ex
If $p=1$ or $p=\infty$, the computation can be vectorized, with the
per-lane operations
$${\rm add\_abs}(x,y)=|x|+|y|,\qquad{\rm max\_abs}(x,y)=\max\{|x|,|y|\},$$
respectively, instead of ${\rm hypot}(x,y)$ from the paper.
\bye
