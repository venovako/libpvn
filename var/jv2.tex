\documentclass[a4paper,12pt,twoside]{article}
\usepackage{amsmath}
\usepackage{amssymb}
\usepackage{url}
\title{Complex hyperbolic $2\times 2$ rotations}
\author{Vedran Novakovi\'{c}\footnote{with suggestions from Vjeran Hari}}
\begin{document}
\maketitle
Let $V\in\mathbb{C}^{2\times 2}$ be $J$-unitary and let
$A\in\mathbb{C}^{2\times 2}$ be a Hermitian positive semidefinite
matrix, where
\begin{equation}
  A=\begin{bmatrix}
  a_{11} & \overline{a_{21}}\\
  a_{21} & a_{22}
  \end{bmatrix},\quad
  J=\begin{bmatrix}
  1 & \hphantom{-}0\\
  0 & -1
  \end{bmatrix},\quad
  V=\begin{bmatrix}
  \hphantom{\mathrm{e}^{\mathrm{i}\beta}}\cosh\phi & \mathrm{e}^{-\mathrm{i}\beta}\sinh\phi\\
  \mathrm{e}^{\mathrm{i}\beta}\sinh\phi & \hphantom{\mathrm{e}^{-\mathrm{i}\beta}}\cosh\phi
  \end{bmatrix}.
  \label{e:1}
\end{equation}
Then, $V^{\ast}=V$ and $V^{\ast}JV=VJV^{\ast}=J$.  Find $\beta$ and
$\phi$ such that $V^{\ast}AV=\Xi$,
\begin{equation}
  \Xi\!=\!\begin{bmatrix}
  \xi_1 & \!\!0\\
  0 & \!\!\xi_2
  \end{bmatrix}\!=\!\begin{bmatrix}
  \hphantom{\mathrm{e}^{\mathrm{i}\beta}}\cosh\phi & \!\mathrm{e}^{-\mathrm{i}\beta}\sinh\phi\\
  \mathrm{e}^{\mathrm{i}\beta}\sinh\phi & \!\hphantom{\mathrm{e}^{-\mathrm{i}\beta}}\cosh\phi
  \end{bmatrix}\!\!\begin{bmatrix}
  a_{11} & \!\overline{a_{21}}\\
  a_{21} & \!a_{22}
  \end{bmatrix}\!\!\begin{bmatrix}
  \hphantom{\mathrm{e}^{\mathrm{i}\beta}}\cosh\phi & \!\mathrm{e}^{-\mathrm{i}\beta}\sinh\phi\\
  \mathrm{e}^{\mathrm{i}\beta}\sinh\phi & \!\hphantom{\mathrm{e}^{-\mathrm{i}\beta}}\cosh\phi
  \end{bmatrix}.
  \label{e:2}
\end{equation}
Dividing~\eqref{e:2} by $\cosh^2\phi>0$ gives, with
$\xi_i'=\xi_i^{}/\cosh^2\phi$,
\begin{equation}
  \begin{bmatrix}
  \hphantom{\mathrm{e}^{\mathrm{i}\beta}}1 & \!\mathrm{e}^{-\mathrm{i}\beta}\tanh\phi\\
  \mathrm{e}^{\mathrm{i}\beta}\tanh\phi & \!\hphantom{\mathrm{e}^{-\mathrm{i}\beta}}1
  \end{bmatrix}\!\!\begin{bmatrix}
  a_{11}^{} & \!\overline{a_{21}^{}}\\
  a_{21}^{} & \!a_{22}^{}
  \end{bmatrix}\!\!\begin{bmatrix}
  \hphantom{\mathrm{e}^{\mathrm{i}\beta}}1 & \!\mathrm{e}^{-\mathrm{i}\beta}\tanh\phi\\
  \mathrm{e}^{\mathrm{i}\beta}\tanh\phi & \!\hphantom{\mathrm{e}^{-\mathrm{i}\beta}}1
  \end{bmatrix}\!=\!\begin{bmatrix}
  \xi_1' & \!0\\
  0 & \!\xi_2'
  \end{bmatrix}.
  \label{e:3}
\end{equation}
Multiplying the first two matrices in~\eqref{e:3} gives
\begin{displaymath}
  \Xi'=\begin{bmatrix}
  a_{11}^{}+a_{21}^{}\mathrm{e}^{-\mathrm{i}\beta}\tanh\phi & \overline{a_{21}^{}}+a_{22}^{}\mathrm{e}^{-\mathrm{i}\beta}\tanh\phi\\
  a_{11}^{}\mathrm{e}^{\mathrm{i}\beta}\tanh\phi+a_{21}^{} & \overline{a_{21}^{}}\mathrm{e}^{\mathrm{i}\beta}\tanh\phi+a_{22}^{}
  \end{bmatrix}\!\!\begin{bmatrix}
  \hphantom{\mathrm{e}^{\mathrm{i}\beta}}1 & \!\mathrm{e}^{-\mathrm{i}\beta}\tanh\phi\\
  \mathrm{e}^{\mathrm{i}\beta}\tanh\phi & \!\hphantom{\mathrm{e}^{-\mathrm{i}\beta}}1
  \end{bmatrix},
\end{displaymath}
with the final multiplication producing, elementwise,
\begin{eqnarray}
    \xi_1'=&a_{11}^{}+(2\Re(a_{21}^{}\mathrm{e}^{-\mathrm{i}\beta})+a_{22}^{}\tanh\phi)\tanh\phi,\label{e:4}\\
    0=&\overline{a_{21}^{}}+(a_{11}^{}+a_{22}^{}+a_{21}^{}\mathrm{e}^{-\mathrm{i}\beta}\tanh\phi)\mathrm{e}^{-\mathrm{i}\beta}\tanh\phi,\label{e:5}\\
    0=&a_{21}^{}+(a_{11}^{}+a_{22}^{}+\overline{a_{21}^{}}\mathrm{e}^{\mathrm{i}\beta}\tanh\phi)\mathrm{e}^{\mathrm{i}\beta}\tanh\phi,\label{e:6}\\
    \xi_2'=&a_{22}^{}+(2\Re(a_{21}^{}\mathrm{e}^{-\mathrm{i}\beta})+a_{11}^{}\tanh\phi)\tanh\phi,\label{e:7}
\end{eqnarray}
since $z+\bar{z}=2\Re{z}$.  Evidently, $\xi_i\in\mathbb{R}$.  If
$a_{21}=0$ then $\tanh\phi=0$, and vice versa, if $\tanh\phi=0$, then
from~\eqref{e:5} (or~\eqref{e:6}, which is the complex conjugate
of~\eqref{e:5}) it follows $a_{21}=0$.  Therefore, assume in the
following that $\tanh\phi\ne 0$.

Multiplying~\eqref{e:5} by $\mathrm{e}^{\mathrm{i}\beta}$
and~\eqref{e:6} by $\mathrm{e}^{-\mathrm{i}\beta}$, it follows
\begin{eqnarray}
  0=&\overline{a_{21}^{}}\mathrm{e}^{\mathrm{i}\beta}+(a_{11}^{}+a_{22}^{})\tanh\phi+a_{21}^{}\mathrm{e}^{-\mathrm{i}\beta}\tanh^2\phi,\label{e:8}\\
  0=&a_{21}^{}\mathrm{e}^{-\mathrm{i}\beta}+(a_{11}^{}+a_{22}^{})\tanh\phi+\overline{a_{21}^{}}\mathrm{e}^{\mathrm{i}\beta}\tanh^2\phi.\label{e:9}
\end{eqnarray}
By extracting the common middle term on the right hand sides of~\eqref{e:8} and~\eqref{e:9},
\begin{eqnarray}
  -(a_{11}^{}+a_{22}^{})\tanh\phi=\overline{a_{21}^{}}\mathrm{e}^{\mathrm{i}\beta}+a_{21}^{}\mathrm{e}^{-\mathrm{i}\beta}\tanh^2\phi,\label{e:10}\\
  -(a_{11}^{}+a_{22}^{})\tanh\phi=a_{21}^{}\mathrm{e}^{-\mathrm{i}\beta}+\overline{a_{21}^{}}\mathrm{e}^{\mathrm{i}\beta}\tanh^2\phi,\label{e:11}
\end{eqnarray}
it follows that the right hand sides of~\eqref{e:10} and~\eqref{e:11}
have to be equal,
\begin{equation}
  \overline{a_{21}^{}}\mathrm{e}^{\mathrm{i}\beta}+a_{21}^{}\mathrm{e}^{-\mathrm{i}\beta}\tanh^2\phi=a_{21}^{}\mathrm{e}^{-\mathrm{i}\beta}+\overline{a_{21}^{}}\mathrm{e}^{\mathrm{i}\beta}\tanh^2\phi,
  \label{e:12}
\end{equation}
while at the same time being the complex conjugates of one another.
Therefore, both sides in~\eqref{e:12} are real, what~\eqref{e:10}
and~\eqref{e:11} also suggest.  With
\begin{equation}
  \beta=\arg(a_{21}),\quad\text{i.e.,}\quad\mathrm{e}^{\mathrm{i}\beta}=\frac{a_{21}}{|a_{21}|},\quad(a_{21}=0\implies\beta=0),
  \label{e:13}
\end{equation}
this condition is always fulfilled.  If $a_{21}$ is real
in~\eqref{e:1}, then
$\mathrm{e}^{\mathrm{i}\beta}=\mathop{\mathrm{sign}}{a_{21}}$.

Now, \eqref{e:8} and~\eqref{e:9} become
\begin{equation}
  |a_{21}|(1+\tanh^2\phi)=-(a_{11}+a_{22})\tanh\phi,
  \label{e:14}
\end{equation}
or, after rearranging~\eqref{e:14},
\begin{equation}
  \frac{-|a_{21}|}{a_{11}+a_{22}}=\frac{\tanh\phi}{1+\tanh^2\phi}=\frac{1}{2}\tanh(2\phi).
  \label{e:15}
\end{equation}
Note that $a_{11}\ge 0$ and $a_{22}\ge 0$ by definition, so
$a_{11}+a_{22}=0$ if and only if $a_{11}=0$ and $a_{22}=0$, in which
case $a_{21}=0$, and thus~\eqref{e:15} does not even have to be
evaluated in this degenerate case to compute $\tanh\phi=0$.

The proper sequence of checks on the inputs $a_{11}$, $a_{22}$, and
$a_{21}$ is thus
\begin{enumerate}
\item if $a_{11}<0$ then ERROR; else
\item if $a_{22}<0$ then ERROR; else
\item if $a_{21}'=0$ then $\tanh\phi=0$,
\end{enumerate}
where $a_{21}'$ is $a_{21}^{}$, scaled as in~\eqref{e:20}.  Else,
proceed to compute $\tanh(2\phi)$.  If it is detected that the
computed $\tanh(2\phi)\le-1$, then ERROR, since the input does not
define a positive semidefinite matrix.

Alternatively, the hyperbolic cotangent of $2\phi$ might be computed
as:
\begin{equation}
  \coth(2\phi)=\frac{a_{11}+a_{22}}{-2|a_{21}|},
  \label{e:16}
\end{equation}
but this can easily cause overflow for small $|a_{21}|$.  On the other
hand, underflow of $\tanh(2\phi)$, as long as it is not exactly zero,
will cause no trouble in the further computation, and might preserve
at least some information, possibly resulting in a non-zero
$\tanh\phi$ as well.

\looseness=-1
There are two solutions for the quadratic equation for $\tanh\phi$
from~\eqref{e:15}, only one of which obeys $|\tanh\phi|<1$,
\begin{equation}
  \tanh\phi=\frac{1-\sqrt{1-\tanh^2(2\phi)}}{\tanh(2\phi)}.
  \label{e:17}
\end{equation}
However, if the numerator and the denominator in~\eqref{e:17} are
multiplied by $1+\sqrt{1-\tanh^2(2\phi)}$, a more stable form emerges,
\begin{equation}
  \begin{aligned}
    \tanh\phi&=\frac{1-\sqrt{1-\tanh^2(2\phi)}}{\tanh(2\phi)}\cdot\frac{1+\sqrt{1-\tanh^2(2\phi)}}{1+\sqrt{1-\tanh^2(2\phi)}}\\
    &=\frac{1-(1-\tanh^2(2\phi))}{\tanh(2\phi)\left(1+\sqrt{1-\tanh^2(2\phi)}\right)}\\
    &=\frac{\tanh(2\phi)}{1+\sqrt{1-\tanh^2(2\phi)}}\approx\frac{\tanh(2\phi)}{1+\sqrt{\mathop{\mathrm{fma}}(-\tanh(2\phi),\tanh(2\phi),1)}}.
  \end{aligned}
  \label{e:18}
\end{equation}
Since $|\tanh(2\phi)|<1$, from~\eqref{e:18} it follows that
$|\tanh\phi|<1$ as well.  Now,
\begin{equation}
  \begin{aligned}
    \cosh\phi&=\frac{1}{\sqrt{1-\tanh^2\phi}}\approx\mathop{\mathrm{rsqrt}}(\mathop{\mathrm{fma}}(-\tanh\phi,\tanh\phi,1)),\\
    \sinh\phi&=\tanh\phi\cdot\cosh\phi.
  \end{aligned}
  \label{e:19}
\end{equation}
If $\tanh\phi$ is regarded as a computed floating-point value, then
$|\tanh\phi|\le 1^-$, where $1^-$ is the first floating-point
predecessor of unity.  Otherwise, the computed $|\tanh(2\phi)|$ would
have to be unity, what has already been ruled out.  Thus,
\begin{displaymath}
  \tanh^2\phi<|\tanh\phi|\le 1^-\implies 1-\tanh^2\phi\ge 1-1^-.
\end{displaymath}
Since $1-1^-$ is a floating-point value in the normal range, its
square root is as well, so $\cosh\phi$ cannot overflow
in~\eqref{e:19}, and neither can $\sinh\phi$.

The input data has to be prescaled by the highest possible power of
two, $2^s$, to get $a_{11}'=2^s a_{11}^{}$, $a_{22}'=2^s a_{22}^{}$,
and $a_{21}'=2^s a_{21}^{}$, such that
\begin{equation}
  \max\{a_{11}',a_{22}'\}\le\nu/2,\quad
  \max\{|\Re{a_{21}'}|,|\Im{a_{21}'}|\}\le\nu/2,
  \label{e:20}
\end{equation}
where $\nu$ is the largest finite floating-point value.  This way no
quantity in~\eqref{e:15} will overflow\footnote{not even $2|a_{21}'|$,
since $A$ is positive semidefinite, so
$2|a_{21}'|\le 2\sqrt{a_{11}'a_{22}'}\le\nu$, but the upper bound of
$\nu/4$ instead of $\nu/2$ in~\eqref{e:20} might be safer}, and the
chances of dealing with subnormal values will be minimized.  See the
scaling of $A$ and the computation of $\mathrm{e}^{\mathrm{i}\beta}$
in:\\\url{https://doi.org/10.1016/j.cam.2024.116003}
\end{document}
